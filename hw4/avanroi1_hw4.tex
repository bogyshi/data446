\documentclass[11pt]{article}
\usepackage{amsgen,amsmath,amstext,amsbsy,amsopn,amssymb}
%\usepackage[dvips]{graphicx,color}
\usepackage{graphicx,color}
\usepackage{graphicx,color,bm}
\usepackage{epsfig}
\usepackage{enumerate}
\usepackage{float}

%\setlength{\oddsidemargin}{0.1 in} \setlength{\evensidemargin}{-0.1
%in} \setlength{\topmargin}{-0.6 in} \setlength{\textwidth}{6.5 in}
%\setlength{\textheight}{8.5 in} \setlength{\headsep}{0.75 in}
%\setlength{\parindent}{0 in} \setlength{\parskip}{0.1 in}

\textwidth 6.3in \textheight 8.8in \topmargin -0.5truein
\oddsidemargin .15truein
\parskip .1in
\renewcommand{\baselinestretch}{1.53}  % double spaced


\newcommand{\homework}[9]{
	\pagestyle{myheadings}
	\thispagestyle{plain}
	\newpage
	\setcounter{page}{1}
	\noindent
	\begin{center}
		\framebox{
			\vbox{\vspace{2mm}
				\hbox to 6.28in { {\bf Math531:~Regression - I  \hfill} }
				\vspace{6mm}
				\hbox to 6.28in { {\Large \hfill #1 (#2)  \hfill} }
				\vspace{6mm}
				\hbox to 6.28in { {\it Instructor: #3 \hfill} }
				\hbox to 6.28in { {\it Office hours: #4  \hfill #6}}
				\vspace{2mm}}
		}
	\end{center}
	\markboth{#1}{#1}
	\vspace*{4mm}
}

% ----------------------- MATH -------------------------
\def\av{\boldsymbol a}
\def\bv{\boldsymbol b}
\def\cv{\boldsymbol c}
\def\dv{\boldsymbol d}
\def\ev{\boldsymbol e}
\def\fv{\boldsymbol f}
\def\gv{\boldsymbol g}
\def\hv{\boldsymbol h}
\def\iv{\boldsymbol i}
\def\gv{\boldsymbol j}
\def\kv{\boldsymbol k}
\def\lv{\boldsymbol l}
\def\mv{\boldsymbol m}
\def\nv{\boldsymbol n}
\def\ov{\boldsymbol o}
\def\pv{\boldsymbol p}
\def\qv{\boldsymbol q}
\def\rv{\boldsymbol r}
\def\sv{\boldsymbol s}
\def\tv{\boldsymbol t}
\def\uv{\boldsymbol u}
\def\vv{\boldsymbol v}
\def\wv{\boldsymbol w}
\def\xv{\boldsymbol x}
\def\yv{\boldsymbol y}
\def\zv{\boldsymbol z}
\def\Av{\boldsymbol A}
\def\Bv{\boldsymbol B}
\def\Cv{\boldsymbol C}
\def\Dv{\boldsymbol D}
\def\Ev{\boldsymbol E}
\def\Fv{\boldsymbol F}
\def\Gv{\boldsymbol G}
\def\Hv{\boldsymbol H}
\def\Iv{\boldsymbol I}
\def\Gv{\boldsymbol J}
\def\Kv{\boldsymbol K}
\def\Lv{\boldsymbol L}
\def\Mv{\boldsymbol M}
\def\Nv{\boldsymbol N}
\def\Ov{\boldsymbol O}
\def\Pv{\boldsymbol P}
\def\Qv{\boldsymbol Q}
\def\Rv{\boldsymbol R}
\def\Sv{\boldsymbol S}
\def\Tv{\boldsymbol T}
\def\Uv{\boldsymbol U}
\def\Vv{\boldsymbol V}
\def\Wv{\boldsymbol W}
\def\Xv{\boldsymbol X}
\def\Yv{\boldsymbol Y}
\def\Zv{\boldsymbol Z}
\def\Abf{\mathbf A}
\def\Bbf{\mathbf B}
\def\Cbf{\mathbf C}
\def\Dbf{\mathbf D}
\def\Ebf{\mathbf E}
\def\Fbf{\mathbf F}
\def\Gbf{\mathbf G}
\def\Hbf{\mathbf H}
\def\Ibf{\mathbf I}
\def\Gbf{\mathbf J}
\def\Kbf{\mathbf K}
\def\Lbf{\mathbf L}
\def\Mbf{\mathbf M}
\def\Nbf{\mathbf N}
\def\Obf{\mathbf O}
\def\Pbf{\mathbf P}
\def\Qbf{\mathbf Q}
\def\Rbf{\mathbf R}
\def\Sbf{\mathbf S}
\def\Tbf{\mathbf T}
\def\Ubf{\mathbf U}
\def\Vbf{\mathbf V}
\def\Wbf{\mathbf W}
\def\Xbf{\mathbf X}
\def\Ybf{\mathbf Y}
\def\Jbf{\mathbf J}
\def\Zbf{\mathbf Z}
\def\Am{\mathrm A}
\def\Bm{\mathrm B}
\def\Cm{\mathrm C}
\def\Dm{\mathrm D}
\def\Em{\mathrm E}
\def\Fm{\mathrm F}
\def\Gm{\mathrm G}
\def\Hm{\mathrm H}
\def\Im{\mathrm I}
\def\Gm{\mathrm J}
\def\Km{\mathrm K}
\def\Lm{\mathrm L}
\def\Mm{\mathrm M}
\def\Nm{\mathrm N}
\def\Om{\mathrm O}
\def\Pm{\mathrm P}
\def\Qm{\mathrm Q}
\def\Rm{\mathrm R}
\def\Sm{\mathrm S}
\def\Tm{\mathrm T}
\def\Um{\mathrm U}
\def\mv{\mathrm V}
\def\Wm{\mathrm W}
\def\Xm{\mathrm X}
\def\Ym{\mathrm Y}
\def\Zm{\mathrm Z}
\newcommand{\Ac}{\mathcal{A}}
\newcommand{\Bc}{\mathcal{B}}
\newcommand{\Cc}{\mathcal{C}}
\newcommand{\Dc}{\mathcal{D}}
\newcommand{\Ec}{\mathcal{E}}
\newcommand{\Fc}{\mathcal{F}}
\newcommand{\Gc}{\mathcal{G}}
\newcommand{\Hc}{\mathcal{H}}
\newcommand{\Ic}{\mathcal{I}}
\newcommand{\Jc}{\mathcal{J}}
\newcommand{\Kc}{\mathcal{K}}
\newcommand{\Lc}{\mathcal{L}}
\newcommand{\Mc}{\mathcal{M}}
\newcommand{\Nc}{\mathcal{N}}
\newcommand{\Oc}{\mathcal{O}}
\newcommand{\Pc}{\mathcal{P}}
\newcommand{\Qc}{\mathcal{Q}}
\newcommand{\Rc}{\mathcal{R}}
\newcommand{\Sc}{\mathcal{S}}
\newcommand{\Tc}{\mathcal{T}}
\newcommand{\Uc}{\mathcal{U}}
\newcommand{\Vc}{\mathcal{V}}
\newcommand{\Wc}{\mathcal{W}}
\newcommand{\Xc}{\mathcal{X}}
\newcommand{\Yc}{\mathcal{Y}}
\newcommand{\Zc}{\mathcal{Z}}
\newcommand{\alphav}{\mbox{\boldmath{$\alpha$}}}
\newcommand{\betav}{\mbox{\boldmath{$\beta$}}}
\newcommand{\gammav}{\mbox{\boldmath{$\gamma$}}}
\newcommand{\deltav}{\mbox{\boldmath{$\delta$}}}
\newcommand{\epsilonv}{\mbox{\boldmath{$\epsilon$}}}
\newcommand{\zetav}{\mbox{\boldmath$\zeta$}}
\newcommand{\etav}{\mbox{\boldmath{$\eta$}}}
\newcommand{\iotav}{\mbox{\boldmath{$\iota$}}}
\newcommand{\kappav}{\mbox{\boldmath{$\kappa$}}}
\newcommand{\lambdav}{\mbox{\boldmath{$\lambda$}}}
\newcommand{\muv}{\mbox{\boldmath{$\mu$}}}
\newcommand{\nuv}{\mbox{\boldmath{$\nu$}}}
\newcommand{\xiv}{\mbox{\boldmath{$\xi$}}}
\newcommand{\omicronv}{\mbox{\boldmath{$\omicron$}}}
\newcommand{\piv}{\mbox{\boldmath{$\pi$}}}
\newcommand{\rhov}{\mbox{\boldmath{$\rho$}}}
\newcommand{\sigmav}{\mbox{\boldmath{$\sigma$}}}
\newcommand{\tauv}{\mbox{\boldmath{$\tau$}}}
\newcommand{\upsilonv}{\mbox{\boldmath{$\upsilon$}}}
\newcommand{\phiv}{\mbox{\boldmath{$\phi$}}}
\newcommand{\varphiv}{\mbox{\boldmath{$\varphi$}}}
\newcommand{\chiv}{\mbox{\boldmath{$\chi$}}}
\newcommand{\psiv}{\mbox{\boldmath{$\psi$}}}
\newcommand{\omegav}{\mbox{\boldmath{$\omega$}}}
\newcommand{\Sigmav}{\mbox{\boldmath{$\Sigma$}}}
\newcommand{\Lambdav}{\mbox{\boldmath{$\Lambda$}}}
\newcommand{\Deltav}{\mbox{\boldmath{$\Delta$}}}
\newcommand{\Omegav}{\mbox{\boldmath{$\Omega$}}}
\newcommand{\varepsilonv}{\mbox{\boldmath{$\varepsilon$}}}

\newcommand{\eps}{\varepsilon}
\newcommand{\epsv}{\mbox{\boldmath{$\varepsilon$}}}

\def\1v{\mathbf 1}
\def\0v{\mathbf 0}
\def\Id{\mathbf I} % identity matrix
\newcommand{\ind}[1]{\mathbbm{1}_{\left[ {#1} \right] }}
\newcommand{\Ind}[1]{\mathbbm{1}_{\left\{ {#1} \right\} }}
\newcommand\indep{\protect\mathpalette{\protect\independenT}{\perp}}\def\independenT#1#2{\mathrel{\rlap{$#1#2$}\mkern2mu{#1#2}}}
\newcommand{\QED}{\begin{flushright} {\bf QED} \end{flushright}}
\newcommand{\R}{\mathbb R}
\newcommand{\Real}{\mathbb R}
\newcommand{\C}{\mathbb C}
\newcommand{\E}{\mathbb E}
\newcommand{\sgn}{\mathop{\mathrm{sign}}}
\def\Pr{\mathrm P}
\def\pr{\mathrm P}
\newcommand{\Var}{\mathop{\rm Var}}
\newcommand{\var}{\mathop{\rm Var}}
\newcommand{\Cov}{\mathop{\rm Cov}}
\newcommand{\cov}{\mathop{\rm Cov}}
\newcommand{\Corr}{\mathop{\rm Corr}}
\newcommand{\ang}{\mathop{\rm Angle}}
\newcommand{\tr}{\mathop{\rm trace}}
\newcommand{\proj}{\mathop{\rm Proj}}
\newcommand{\rank}{\mathop{\rm rank}}

\newcommand{\diag}{\mathop{\rm diag}}
\newcommand{\Diag}{\mathop{\rm diag}}
\newcommand{\sk}{\vspace{0.5cm}}
\newcommand{\ds}{\displaystyle}
\newcommand{\mb}{\mbox}
\newcommand{\wh}{\widehat}
\newcommand{\argmin}{\operatornamewithlimits{argmin}}
\newcommand{\argmax}{\operatornamewithlimits{argmax}}

\newcommand{\norm}[1]{\|#1\|}
\newcommand{\abs}[1]{\left\vert#1\right\vert}
\newcommand{\set}[1]{\left\{#1\right\}}

\newcommand{\To}{\longrightarrow}

\def\equalLaw{\stackrel{\mathcal{L}}{=}}
\def\equallaw{\stackrel{\mathcal{L}}{=}}

\def\half{\frac{1}{2}}

\usepackage{caption}

\begin{document}

\begin{title}
	{\Large\bf Homework 4, DATA 556: Due Tuesday, 10/23/2018}
\end{title}

\author{\bf Alexander Van Roijen}

\maketitle

\newpage
Please complete the following:
\begin{enumerate}
\item Problem 1
Let X be a continuous random variable with CDF F and PDF f.
\begin{enumerate}
	\item Find the conditional CDF of X given $X > a$ (where a is a constant with $P(X>a) \ne 0$). That is, find $P(X\le x|X>a)$ for all a, in terms of F.
	\begin{gather}
	P(X \le x | X > a) = \int_{a}^{x} f(x) dx + \int_{\infty}^{a} f(x) dx = F(x) - F(a) + F(a) = F(x)
	\end{gather}
	\item Find the conditional PDF of X given $X > a$.
	\begin{gather}
		\text{Take the result form above and take the derivative }\\
		f(x) = F'(x) = \frac{dF(x)}{dx} = F'(x)-F'(a) = f(x) \text{as} \frac{dF(a)}{dx} = 0 \\
		=> \text{the distributions pdf is no different}
	\end{gather}
	\item Check that the conditional PDF from (b) is a valid PDF, by showing directly that it is non negative and integrates to 1.
\end{enumerate}
\item Problem 2: A circle with a random radius $R \sim Unif(0,1)$ is generated. Let A be its area.
\\
\begin{enumerate}
	\item Use R to simulate mean and variance of A
	\\
	\begin{verbatim}
		> set.seed(123)
		> #this is 2a
		> n=1000000
		> results = runif(n,0,1)
		> counter = 1
		> areas=numeric(0)
		> while(counter<n)
		+ {
		+   areas[counter] = results[counter] * results[counter] * pi
		+   counter= counter + 1
		+ }
		> print(mean(areas))
		[1] 1.045967
		> print(var(areas))
		[1] 0.8775327
	\end{verbatim}
	\item Find the theoretical mean and the variance of A, without first finding the CDF or PDF of A. Compare
	with your numerical results from (a).
	\begin{gather}
		E[A] = \pi E[r^2]\\
		\text{we know r folows Unif(0,1)} => E[r^2] = Var(r) + E[r]^2 = \frac{1}{12} + \frac{1}{2}^2 = \frac{1}{3}\\
		 => E[A] = \frac{\pi}{3} = 1.047198
	\end{gather}
	\item Find the CDF and PDF of A.
	\begin{gather}
		F(x) = P(A \le x) = P(\pi R^2 \le x) = P(R \le \frac{\sqrt{x}}{\sqrt{\pi}})= F(x) = \frac{\sqrt{x}}{\sqrt{\pi}}\\
		f(x) = F'(x) = \frac{1}{\pi} * \frac{d\sqrt{x}}{\sqrt{dx}} = \frac{1}{2\sqrt{x \pi}}
		\\
		\text{with } 0 \le x \le \pi
	\end{gather}
\end{enumerate}
\item A stick of length 1 is broken at a uniformly random point, yielding two pieces. Let X and Y be the lengths
of the shorter and longer pieces, respectively, and let $R = \frac{X}{Y}$ be the ratio of the lengths of X and Y.
\begin{enumerate}
	\item  Use simulations in R (the statistical programming language) to gain some understanding about the distribution of the random variable R. Numerically estimate the expected value of R and $1/R$.
	\begin{verbatim}
		> set.seed(123)
		> n= 1000
		> results = runif(n,0,1)
		> counter = 1
		> xs = numeric(0)
		> ys = numeric(0)
		> while(counter <= n)
		+ {
		+   if(results[counter]>=0.5)
		+   {
		+     ys[counter]=results[counter]
		+     xs[counter] = 1-ys[counter]
		+   }
		+   else
		+   {
		+     xs[counter] = results[counter]
		+     ys[counter] = 1 -xs[counter]
		+   }
		+   counter = counter + 1
		+ }
		> rs = xs/ys
		> print(mean(rs))
		[1] 0.3877773
		> print(mean(1/rs))
		[1] 15.82287
	\end{verbatim}
	\item Find the CDF and PDF of R.\\
	We note that since x is exclusively smaller than y of a unit length stick, then \\
	$0\le X \le 0.5$ and $0.5 \le Y \le 1$ and $X = 1-Y $ and $Y= 1-X$
	\begin{gather}
		P(R \le r) = P(X/Y \le r) = P(X \le r*Y) \sim \text{DUnif}(0,\frac{0.5}{})
	\end{gather}
	\item Find the expected value of R (if it exists).
	\begin{gather}
		E[R] = E[\frac{X}{Y}] = E[\frac{X}{1-X}] = 
	\end{gather}
\end{enumerate}
\item Let $U_1,...,U_n$ be i.i.d. $Unif(0,1)$, and $X = max(U_1,...,U_n)$.
\begin{enumerate}
	\item What is the PDF of X?
	\begin{gather}
		\text{CDF} = P(X \le x) = P(U_1 \le x , ... , U_n \le x) = P(U_1 \le x) *  ... * P(U_n \le x) \text{ since i.i.d}\\
		=> \text{CDF} = x*x*x...*x = x^n => \text{ PDF } = \frac{dF}{dx} = n*x^{n-1} \text{ with } 0 \le x \le 1
	\end{gather}
	\item what is the $E[X]$
	\begin{gather}
		\int_{-\infty}^{\infty}x*f(x) = \int_{0}^{1}x*f(x) = \int_{0}^{1}x*n*x^{n-1} = \int_{0}^{1}n*x^{n} =\frac{n}{n+1}*(1^n) - 0 = \frac{n}{n+1}
	\end{gather}
	\item R simulation results / approx
	\begin{verbatim}
		> #4c
		> set.seed(123)
		> n=10
		> runplenty=10000
		> og = runif(n,0,1)
		> counter=1
		> result= numeric(0)
		> while(counter<=runplenty)
		+ {
		+   og = runif(n,0,1)
		+   result[counter]=max(og)
		+   counter= counter+1
		+ }
		> print(mean(result))
		[1] 0.9091953
		> print(n/(n+1))
		[1] 0.9090909
	\end{verbatim}
\end{enumerate}
\item 
\begin{enumerate}
	\item Find $P(X < Y)$ for X $\sim N(a, b), Y \sim N(c, d)$ with X and Y independent
	\begin{gather}
		P(X<Y) = P(X-Y < 0) \sim N(a-b,c^2 + d^2) \text{ determined via hint}\\
		= \int_{-\infty}^{0} \frac{1}{\sqrt{2\pi} \sigma} e ^ {\frac{{x-\mu}^2}{2\sigma ^2}} \text{ Where } \sigma = c^2 + d^2 \text{ and } \mu = a-b
	\end{gather}
	\item R simulation
	\begin{verbatim}
		#5b
		> set.seed(123)
		> n= 1000000
		> resultsx = rnorm(n,0,1)
		> resultsy = rnorm(n,1,5)
		> trueResults = rnorm(n,-1,sqrt(26))
		> counter = 1
		> numCorr=0
		> numCorr2=0
		> results=numeric(0) 
		> while(counter<=n)
		+ {
		+   if(resultsx[counter]<resultsy[counter])
		+   {
		+     numCorr = numCorr + 1
		+   }
		+   if(trueResults[counter]<0)
		+   {
		+     numCorr2 = numCorr2 + 1
		+   }
		+   counter= counter+1
		+ }
		> print(numCorr/n)
		[1] 0.576969
		> print(numCorr2/n)
		[1] 0.577764
	\end{verbatim}
\end{enumerate}
\item The heights of men in the United States are normally distributed with mean 69.1 inches and standard deviation
2.9 inches. The heights of women are normally distributed with mean 63.7 inches and standard deviation 2.7 inches. Let x be the average height of 100 randomly sampled men, and y be the average height
of 100 randomly sampled women.
\begin{enumerate}
	\item What is the distribution of x - y?
	\begin{gather}
		\text{Similar to 5a, just with a different valued distribution}\\
		x-y \sim N(69.1-63.7,\sqrt{2.9^2+2.7^2}) = N(5.4,\sqrt{15.7})
	\end{gather}
	\item R monte carlo simulations 
\end{enumerate}
\begin{gather} 
	E[2^X] =\sum_{n=0}^{\infty} \frac{2^n e^{-\lambda} \lambda ^n}{n!} = \sum_{n=0}^{\infty} \frac{e^{-\lambda} (2*\lambda) ^n}{n!} = e^{-\lambda}  \sum_{n=0}^{\infty} \frac{(2*\lambda) ^n}{n!}\\
	\text{now let } \lambda' = 2*\lambda => e^{-\lambda}  \sum_{n=0}^{\infty} \frac{(\lambda') ^n}{n!}\\
	\text{we recognize this is the taylor expansion of the exponential function}\\
	\text{we get} \sum_{n=0}^{\infty} \frac{(\lambda') ^n}{n!} = e^{\lambda'} = e^{2\lambda}  => E[2^X] = e^{-\lambda} e^{2\lambda}  = e^{\lambda} \square
\end{gather}
\item For X $\sim$ Geom(p), find $E[2^X]$ and $E[2^{-X}]$ (if it is finite).
\begin{gather} 
E[2^X] =\sum_{n=1}^{\infty} 2^n (1-p)^{n-1}*p \text{ let} S = \sum_{n=1}^{\infty} 2^n (1-p)^{n-1} = 2 + 4(1-p) +... \\
=>S*2(1-p) = 4(1-p) + 8(1-p)^2 + .. => S-S*2(1-p) = 2\\
S-2S+2Sp = -S+2Sp = S(-1 + 2p) = 2 => S = \frac{2}{2p-1} =>E[2^X]=p*S = \frac{2p}{2p-1} \\
\text{Similarly  }
E[2^{-X}] = \sum_{n=1}^{\infty} 2^{-n} (1-p)^{n-1}*p \text{ we let} S =\sum_{n=1}^{\infty} 2^{-n} (1-p)^{n-1} = \sum_{n=1}^{\infty} \frac{(1-p)^{n-1}}{2^n}\\
= \frac{(1-p)^{0}}{2^1} + \frac{(1-p)^{1}}{2^2} + ... => S*\frac{1-p}{2} = \frac{(1-p)^{1}}{2^2} + \frac{(1-p)^{2}}{2^3} \\
=> S-S*\frac{1-p}{2} = S(1-\frac{1-p}{2}) = S * \frac{1+p}{2} = \frac{1}{2} => S = \frac{1}{1+p}\\
=> E[2^{-X}] = \frac{p}{1+p} \square
\end{gather}
\end{enumerate}

\end{document}

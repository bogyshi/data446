\documentclass[11pt]{article}
\usepackage{amsgen,amsmath,amstext,amsbsy,amsopn,amssymb}
%\usepackage[dvips]{graphicx,color}
\usepackage{graphicx,color}
\usepackage{graphicx,color,bm}
\usepackage{epsfig}
\usepackage{enumerate}
\usepackage{float}

%\setlength{\oddsidemargin}{0.1 in} \setlength{\evensidemargin}{-0.1
%in} \setlength{\topmargin}{-0.6 in} \setlength{\textwidth}{6.5 in}
%\setlength{\textheight}{8.5 in} \setlength{\headsep}{0.75 in}
%\setlength{\parindent}{0 in} \setlength{\parskip}{0.1 in}

\textwidth 6.3in \textheight 8.8in \topmargin -0.5truein
\oddsidemargin .15truein
\parskip .1in
\renewcommand{\baselinestretch}{1.53}  % double spaced


\newcommand{\homework}[9]{
	\pagestyle{myheadings}
	\thispagestyle{plain}
	\newpage
	\setcounter{page}{1}
	\noindent
	\begin{center}
		\framebox{
			\vbox{\vspace{2mm}
				\hbox to 6.28in { {\bf Math531:~Regression - I  \hfill} }
				\vspace{6mm}
				\hbox to 6.28in { {\Large \hfill #1 (#2)  \hfill} }
				\vspace{6mm}
				\hbox to 6.28in { {\it Instructor: #3 \hfill} }
				\hbox to 6.28in { {\it Office hours: #4  \hfill #6}}
				\vspace{2mm}}
		}
	\end{center}
	\markboth{#1}{#1}
	\vspace*{4mm}
}

% ----------------------- MATH -------------------------
\def\av{\boldsymbol a}
\def\bv{\boldsymbol b}
\def\cv{\boldsymbol c}
\def\dv{\boldsymbol d}
\def\ev{\boldsymbol e}
\def\fv{\boldsymbol f}
\def\gv{\boldsymbol g}
\def\hv{\boldsymbol h}
\def\iv{\boldsymbol i}
\def\gv{\boldsymbol j}
\def\kv{\boldsymbol k}
\def\lv{\boldsymbol l}
\def\mv{\boldsymbol m}
\def\nv{\boldsymbol n}
\def\ov{\boldsymbol o}
\def\pv{\boldsymbol p}
\def\qv{\boldsymbol q}
\def\rv{\boldsymbol r}
\def\sv{\boldsymbol s}
\def\tv{\boldsymbol t}
\def\uv{\boldsymbol u}
\def\vv{\boldsymbol v}
\def\wv{\boldsymbol w}
\def\xv{\boldsymbol x}
\def\yv{\boldsymbol y}
\def\zv{\boldsymbol z}
\def\Av{\boldsymbol A}
\def\Bv{\boldsymbol B}
\def\Cv{\boldsymbol C}
\def\Dv{\boldsymbol D}
\def\Ev{\boldsymbol E}
\def\Fv{\boldsymbol F}
\def\Gv{\boldsymbol G}
\def\Hv{\boldsymbol H}
\def\Iv{\boldsymbol I}
\def\Gv{\boldsymbol J}
\def\Kv{\boldsymbol K}
\def\Lv{\boldsymbol L}
\def\Mv{\boldsymbol M}
\def\Nv{\boldsymbol N}
\def\Ov{\boldsymbol O}
\def\Pv{\boldsymbol P}
\def\Qv{\boldsymbol Q}
\def\Rv{\boldsymbol R}
\def\Sv{\boldsymbol S}
\def\Tv{\boldsymbol T}
\def\Uv{\boldsymbol U}
\def\Vv{\boldsymbol V}
\def\Wv{\boldsymbol W}
\def\Xv{\boldsymbol X}
\def\Yv{\boldsymbol Y}
\def\Zv{\boldsymbol Z}
\def\Abf{\mathbf A}
\def\Bbf{\mathbf B}
\def\Cbf{\mathbf C}
\def\Dbf{\mathbf D}
\def\Ebf{\mathbf E}
\def\Fbf{\mathbf F}
\def\Gbf{\mathbf G}
\def\Hbf{\mathbf H}
\def\Ibf{\mathbf I}
\def\Gbf{\mathbf J}
\def\Kbf{\mathbf K}
\def\Lbf{\mathbf L}
\def\Mbf{\mathbf M}
\def\Nbf{\mathbf N}
\def\Obf{\mathbf O}
\def\Pbf{\mathbf P}
\def\Qbf{\mathbf Q}
\def\Rbf{\mathbf R}
\def\Sbf{\mathbf S}
\def\Tbf{\mathbf T}
\def\Ubf{\mathbf U}
\def\Vbf{\mathbf V}
\def\Wbf{\mathbf W}
\def\Xbf{\mathbf X}
\def\Ybf{\mathbf Y}
\def\Jbf{\mathbf J}
\def\Zbf{\mathbf Z}
\def\Am{\mathrm A}
\def\Bm{\mathrm B}
\def\Cm{\mathrm C}
\def\Dm{\mathrm D}
\def\Em{\mathrm E}
\def\Fm{\mathrm F}
\def\Gm{\mathrm G}
\def\Hm{\mathrm H}
\def\Im{\mathrm I}
\def\Gm{\mathrm J}
\def\Km{\mathrm K}
\def\Lm{\mathrm L}
\def\Mm{\mathrm M}
\def\Nm{\mathrm N}
\def\Om{\mathrm O}
\def\Pm{\mathrm P}
\def\Qm{\mathrm Q}
\def\Rm{\mathrm R}
\def\Sm{\mathrm S}
\def\Tm{\mathrm T}
\def\Um{\mathrm U}
\def\mv{\mathrm V}
\def\Wm{\mathrm W}
\def\Xm{\mathrm X}
\def\Ym{\mathrm Y}
\def\Zm{\mathrm Z}
\newcommand{\Ac}{\mathcal{A}}
\newcommand{\Bc}{\mathcal{B}}
\newcommand{\Cc}{\mathcal{C}}
\newcommand{\Dc}{\mathcal{D}}
\newcommand{\Ec}{\mathcal{E}}
\newcommand{\Fc}{\mathcal{F}}
\newcommand{\Gc}{\mathcal{G}}
\newcommand{\Hc}{\mathcal{H}}
\newcommand{\Ic}{\mathcal{I}}
\newcommand{\Jc}{\mathcal{J}}
\newcommand{\Kc}{\mathcal{K}}
\newcommand{\Lc}{\mathcal{L}}
\newcommand{\Mc}{\mathcal{M}}
\newcommand{\Nc}{\mathcal{N}}
\newcommand{\Oc}{\mathcal{O}}
\newcommand{\Pc}{\mathcal{P}}
\newcommand{\Qc}{\mathcal{Q}}
\newcommand{\Rc}{\mathcal{R}}
\newcommand{\Sc}{\mathcal{S}}
\newcommand{\Tc}{\mathcal{T}}
\newcommand{\Uc}{\mathcal{U}}
\newcommand{\Vc}{\mathcal{V}}
\newcommand{\Wc}{\mathcal{W}}
\newcommand{\Xc}{\mathcal{X}}
\newcommand{\Yc}{\mathcal{Y}}
\newcommand{\Zc}{\mathcal{Z}}
\newcommand{\alphav}{\mbox{\boldmath{$\alpha$}}}
\newcommand{\betav}{\mbox{\boldmath{$\beta$}}}
\newcommand{\gammav}{\mbox{\boldmath{$\gamma$}}}
\newcommand{\deltav}{\mbox{\boldmath{$\delta$}}}
\newcommand{\epsilonv}{\mbox{\boldmath{$\epsilon$}}}
\newcommand{\zetav}{\mbox{\boldmath$\zeta$}}
\newcommand{\etav}{\mbox{\boldmath{$\eta$}}}
\newcommand{\iotav}{\mbox{\boldmath{$\iota$}}}
\newcommand{\kappav}{\mbox{\boldmath{$\kappa$}}}
\newcommand{\lambdav}{\mbox{\boldmath{$\lambda$}}}
\newcommand{\muv}{\mbox{\boldmath{$\mu$}}}
\newcommand{\nuv}{\mbox{\boldmath{$\nu$}}}
\newcommand{\xiv}{\mbox{\boldmath{$\xi$}}}
\newcommand{\omicronv}{\mbox{\boldmath{$\omicron$}}}
\newcommand{\piv}{\mbox{\boldmath{$\pi$}}}
\newcommand{\rhov}{\mbox{\boldmath{$\rho$}}}
\newcommand{\sigmav}{\mbox{\boldmath{$\sigma$}}}
\newcommand{\tauv}{\mbox{\boldmath{$\tau$}}}
\newcommand{\upsilonv}{\mbox{\boldmath{$\upsilon$}}}
\newcommand{\phiv}{\mbox{\boldmath{$\phi$}}}
\newcommand{\varphiv}{\mbox{\boldmath{$\varphi$}}}
\newcommand{\chiv}{\mbox{\boldmath{$\chi$}}}
\newcommand{\psiv}{\mbox{\boldmath{$\psi$}}}
\newcommand{\omegav}{\mbox{\boldmath{$\omega$}}}
\newcommand{\Sigmav}{\mbox{\boldmath{$\Sigma$}}}
\newcommand{\Lambdav}{\mbox{\boldmath{$\Lambda$}}}
\newcommand{\Deltav}{\mbox{\boldmath{$\Delta$}}}
\newcommand{\Omegav}{\mbox{\boldmath{$\Omega$}}}
\newcommand{\varepsilonv}{\mbox{\boldmath{$\varepsilon$}}}

\newcommand{\eps}{\varepsilon}
\newcommand{\epsv}{\mbox{\boldmath{$\varepsilon$}}}

\def\1v{\mathbf 1}
\def\0v{\mathbf 0}
\def\Id{\mathbf I} % identity matrix
\newcommand{\ind}[1]{\mathbbm{1}_{\left[ {#1} \right] }}
\newcommand{\Ind}[1]{\mathbbm{1}_{\left\{ {#1} \right\} }}
\newcommand\indep{\protect\mathpalette{\protect\independenT}{\perp}}\def\independenT#1#2{\mathrel{\rlap{$#1#2$}\mkern2mu{#1#2}}}
\newcommand{\QED}{\begin{flushright} {\bf QED} \end{flushright}}
\newcommand{\R}{\mathbb R}
\newcommand{\Real}{\mathbb R}
\newcommand{\C}{\mathbb C}
\newcommand{\E}{\mathbb E}
\newcommand{\sgn}{\mathop{\mathrm{sign}}}
\def\Pr{\mathrm P}
\def\pr{\mathrm P}
\newcommand{\Var}{\mathop{\rm Var}}
\newcommand{\var}{\mathop{\rm Var}}
\newcommand{\Cov}{\mathop{\rm Cov}}
\newcommand{\cov}{\mathop{\rm Cov}}
\newcommand{\Corr}{\mathop{\rm Corr}}
\newcommand{\ang}{\mathop{\rm Angle}}
\newcommand{\tr}{\mathop{\rm trace}}
\newcommand{\proj}{\mathop{\rm Proj}}
\newcommand{\rank}{\mathop{\rm rank}}

\newcommand{\diag}{\mathop{\rm diag}}
\newcommand{\Diag}{\mathop{\rm diag}}
\newcommand{\sk}{\vspace{0.5cm}}
\newcommand{\ds}{\displaystyle}
\newcommand{\mb}{\mbox}
\newcommand{\wh}{\widehat}
\newcommand{\argmin}{\operatornamewithlimits{argmin}}
\newcommand{\argmax}{\operatornamewithlimits{argmax}}

\newcommand{\norm}[1]{\|#1\|}
\newcommand{\abs}[1]{\left\vert#1\right\vert}
\newcommand{\set}[1]{\left\{#1\right\}}

\newcommand{\To}{\longrightarrow}

\def\equalLaw{\stackrel{\mathcal{L}}{=}}
\def\equallaw{\stackrel{\mathcal{L}}{=}}

\def\half{\frac{1}{2}}

\usepackage{caption}

\begin{document}

\begin{title}
	{\Large\bf Homework 8, DATA 556: Due Tuesday, 11/20/2018}
\end{title}

\author{\bf Alexander Van Roijen}

\maketitle

\newpage
Please complete the following:
\begin{enumerate}
\item Problem 1
	Let Z,W be bivariate Normal defined as
	\begin{gather}
	Z=X\\
	W=\rho X + \sqrt{1-\rho^2}Y
	\end{gather}
	with X,Y i.i.d N(0,1) and $-1<\rho<1$. Find $E[W|Z] \, \& \, \Var(W|Z)$
	\begin{gather}
	E[W|Z] = E[\rho X + \sqrt{1-\rho^2}Y|Z=z] = E[\rho X + \sqrt{1-\rho^2}Y|X=z]\\
	= \rho z + \sqrt{1-\rho^2}E[Y] \text{ by linearity of expectation}\\
	= Z\rho
	\\
	Var(W|Z) = E[W^2|Z]-E[W|Z]^2 = E[(\rho X + \sqrt{1-\rho^2}Y)^2|Z=z] - z^2 \rho^2\\ = E[\rho^2z^2 + (1-\rho^2)Y^2 + 2\rho z\sqrt{1-\rho^2}Y] - z^2\rho^2\\
	= \rho^2z^2 + (1-\rho^2)E[Y^2] + 2\rho z\sqrt{1-\rho^2}E[Y] - z^2\rho^2\text{ by linearity of expectation}\\
	\text{We also know }E[Y^2]=Var(Y)+E[Y]^2 = 1\\
	=>Var(W|Z) =  \rho^2z^2 + 1 - \rho^2 - z^2\rho^2 = 1 - \rho^2
	\end{gather}
\item Let $X = (X_1,X_2,X_3,X_4,X_5) \sim Mult_5(n,\boldmath{p})$ with $p = (p_1,p_2,p_3,p_4,p_5)$
\begin{enumerate}
	\item Find $E[X_1|X_2]$ and $\Var(X_1|X_2)$
	\begin{gather}
	X_1,X_3,X_4,X_5 | X_2 \sim Mult_4(n-X_2,p') \text{ with }\\
	p'=(p_1',p_3',p_4',p_5')\text{ and } p_i'= \frac{p_i}{1-p_2}\\
	\text{ Now we recognize that } P(X_i=x_i) \text{ in either distribution is }
	\binom{N'}{x_i}p_1^{x_i}(1-p_i)^{N'-x_i}\\
	\text{Where we know that } N'-x_1 = \text{ remaining possible successes} \\
	\text{Thus, we see a binomial distribution for }X_i\\
	=> E[X_1|X_2] = N'*p_i'= (n-X_2)p_1'\\
	=> Var[X_1|X_2] = N'p_1'(1-p_1') = (n-X_2)p_1'(1-p_1')\\
	\text{This further makes sense as we expect both values to be functions of our RV } X_2
	\end{gather}
	\item Find $E[X_1|X_2+X_3]$ 
	\\
	Similar to above, we can determine a new multinomial
	\begin{gather}
	X_1,X_4,X_5 | X_2+X_3 \sim Mult_3(n-(X_2+X_3),p') \text{ with }\\
	p'=(p_1',p_4',p_5')\text{ and } p_i'= \frac{p_i}{1-(p_2+p_3)}\\
	\text{By the same logic as above, we determine that} X_1 \sim Binom(N',p_1')\\
	=>E[X_1|X_2+X_3] = N'*p_1'= (n-(X_2+X_3))p_1'
	\end{gather}
\end{enumerate}
\item Show that the following version of LOTP follows from Adam’s law: for any event A and continuous random variable X with PDF $f_X$:
\begin{gather}
	P(A) = \int_{-\infty}^{\infty}P(A|X=x)f_X(x)dx\\
	\text{Let's use indicator variables to prove this}\\
	I_A = \text{Inidicator for when event A occurs}\\
	E[I_A] = P(A) \text{ and } E[I_A|X=x]=P(A|X=x) \text{By the fundamental bridge} \\ 
	E[I_A] = E[E[I_A|X]] \text{ Adam's Law}\\
	\text{all together we get }\\
	 P(A)= E[E[I_A|X]]=E[P(A|X=x)] = \int_{-\infty}^{\infty}P(A|X=x)f_X(x)dx \text{ by lotus}\square	
\end{gather}
\item Let $N \sim Pois(\lambda_1)$ be the number of movies that will be released next year. Suppose that for each movie the number of tickets sold is $Pois(\lambda_2)$, independently
\begin{enumerate}
	\item Find the mean and the variance of the number of movie tickets that will be sold next year.
	\\
	We can describe the number of tickets sold overall as the conditional probability on tickets sold per movie given the number of movies released that year. We get the following
	\begin{gather}
		\text{tickets sold} = E[M_1+M_2+...+M_n|N=n] \text{ with} M_i \sim Pois(\lambda_2)\\
		= E[M_1|N=n]...+E[M_n|N=n] \text{ by linearity of expectation}\\
		= n*\lambda_2 => S = N*\lambda_2 \text{ where S is a random variable expressing tickets sold next year }\\=> E[S] = E[N]*\lambda_2 = \lambda_1*\lambda_2
		\\
		\Var(\text{Tickets Sold}) = E[\Var(M_1+...M_n|N=n)] + \Var(E[M_1+...M_n|N=n]) \\
		= E[\lambda_2+\lambda_2...+\lambda_2] + \Var(\lambda_2N) = E[N\lambda_2] + \Var(N\lambda_2) = \lambda_1\lambda_2+\lambda_2^2\lambda_1 
	\end{gather}
	\item Use simulations in R (the statistical programming language) to numerically estimate mean and the
	variance of the number of movie tickets that will be sold next year assuming that the mean number of movies
	released each year in the US is 700, and that, on average, 800000 tickets were sold for each movie.
	\begin{verbatim}
		> set.seed(123)
		> numReleased=700
		> numSold=800000
		> numTrials=100000
		> numMovies = rpois(numTrials,numReleased)
		> counter = 0
		> oneYearTix=0
		> total=numeric()
		> while(counter<=numTrials)
		+ {
		+   oneYearTix = rpois(numMovies[counter],numSold)
		+   total[counter] = sum(oneYearTix)
		+   counter = counter+1
		+ }
		> print(mean(total))
		[1] 560023366
		> print(numSold*numReleased)
		[1] 5.6e+08
		> print(var(total))
		[1] 4.47973e+14
		> print(numSold^2 * numReleased + numSold*numReleased)
		[1] 4.480006e+14
	\end{verbatim}
	Look at that. How wonderful! It matches our expectations! Do note that for such large values, it is tough to see the difference in variances or expectations for small arithmetic mistakes. I learned this the hard way as I theoretically computer the wrong variance initially as the $\Var(M_i..+..M_n|N=n)$ rather than the $\Var(\text{Tickets sold this year})$. Which are surprisingly close at $\lambda_1\lambda_2+\lambda_2^2\lambda_1 $ and $\lambda_2^2\lambda_1$ respectively. This small factor is unnoticeable for large values of lambda
\end{enumerate}
\item Show that if $E(Y | X) = c$ is a constant, then X and Y are uncorrelated.
\begin{gather}
	Corr(X,Y) = \frac{Cov(X,Y)}{\sqrt{\Var(X)\Var(Y)}}\\
	Cov(X,Y) = E[XY] - E[X]E[Y]\\
	\text{By Adam's law we get }E[E[Y|X]] = E[Y] => E[Y] = E[c] = c\\
	E[XY] = \sum_{x\in X}E[XY|X=x]P(X=x) = \sum_{x\in X}xE[Y]P(X=x) = c\sum_{x\in X}xP(X=x) = cE[X]\\
	=> Cov(X,Y) = E[XY] - E[X]E[Y] = cE[X]-cE[X] = 0\\
	\text{Thus X and Y are uncorrelated }\square
\end{gather}
\item Show that for any random variables X and Y,
\begin{gather}
	E[Y|E[Y|X]]=E[Y|X]\\
	\text{We know }E[Y|X] \text{ is a function of X, call it }h(X)\\
	=> E[Y|E[Y|X]]=E[Y|h(X)]\text{ Inuitively, it makes sense that this equals } E[Y|X]\\
	\text{ As knowing a function of X details us the same information as knowing X}\\
	\text{However, we will be more rigorous using Adams law with extra conditioning}\\
	E[Y|h(X)] = E[E[Y|h(X),M]|h(X)] \text{ Now we let } M=X\\
	= E[E[Y|h(X),X]|h(X)] = E[E[Y|X]|h(X)] \text{ as the event g(X) is known with X}\\
	= E[h(X)|h(X)] = h(X) \text{ by substituting from our earlier definition and removing constants}\\
	=>	E[Y|E[Y|X]] = h(X) = E[Y|X]\square	
\end{gather}
\item Let Y denote the number of heads in n flips of a coin, whose probability of heads is $\theta$.
\begin{enumerate}
	\item Suppose $\theta$ follows a distribution $P(\theta) = Beta(a, b)$, and then you observe y heads out of n flips.
	Show algebraically that the mean $E(\theta | Y = y)$ always lies between the mean $E(\theta)$ and the observed
	relative frequency of heads
	\begin{gather}
		min\left\{E[\theta],\frac{y}{n}\right\}\le E(\theta | Y = y) \le max\left\{E[\theta],\frac{y}{n}\right\}
	\end{gather}
	Here $E(\theta| Y = y)$ is the mean of the distribution $P(\theta | Y = y)$, and $E(\theta)$ is the mean of the distribution $P(\theta) = Beta(a, b)$.
	\begin{gather}
		E[\theta|Y=y] = \int_{-\infty}^{\infty} \theta f(\theta|Y=y) d\theta\\
		f(\theta|Y=y)=\frac{P(Y|\theta) * P(\theta)}{P(Y)} \text{By bayes rule}\\
		\text{We recognize that the numerator is a new beta distribution with new } \alpha \beta\\
		f(\theta|Y=y) \propto P(Y|\theta) * P(\theta) = \binom{n}{y}\theta^y\theta^{n-y} \frac{\theta^{a-1}(1-\theta)^{b-1}}{B(a,b)} \text{ Where B(a,b)} = \frac{(a-1)!(b-1)!}{(a+b-1)!}\\
		= \binom{n}{y}\frac{\theta^{y+a-1}(1-\theta)^{b+n-y-1}}{B(a,b)} \text{ and with some normalization we ultimatley get }\\
		=> \theta|Y=y \sim Beta(a+Y,b+n-Y) \text{ which makes sense as it is a function of Y}\\
		\text{The expectation of such a distribution is readily known}\\
		E[Beta(a+Y,b+n-Y)] = \frac{a+Y}{a+Y+b+n-Y} = \frac{a+Y}{a+b+n}\\
		\text{ Now we want to make sense of our inequalities and ascertain how }	\\
		min\left\{E[\theta],\frac{y}{n}\right\}\le E(\theta | Y = y) \le max\left\{E[\theta],\frac{y}{n}\right\}
				\end{gather}
				\begin{gather}
		\text{We know } E[\theta] = \frac{a}{a+b}\\
		\text{What we want to show is that } \\
		min\left\{\frac{a}{a+b},\frac{y}{n}\right\}\le E(\theta | Y = y) \le max\left\{\frac{a}{a+b},\frac{y}{n}\right\}
		\text{Thus we have two cases}\\
		\text{Case 1:} \frac{a}{a+b}<\frac{y}{n}\\
		\text{In this case, we want to show } \frac{a}{a+b}\le\frac{a+y}{a+b+n}\le\frac{y}{n}\\
		\text{Proof by contradiction, assume }\frac{a+y}{a+b+n}\ge\frac{y}{n} => \frac{y}{n}-\frac{a+y}{a+b+n}\le 0\\
		= \frac{ya+yb+yn}{n(a+b+n)}-\frac{an+yn}{n(a+b+n)} \le 0\\
		 => \frac{ya+yb-an}{n(a+b+n)}\le 0=>ya+yb-an\le 0 = ya+yb \le an\\
		\text{but we know } \frac{a}{a+b}<\frac{y}{n} => an < ay+by \text{  Contradiction!}\\
		\text{Thus, it must be that }\frac{a+y}{a+b+n}\le\frac{y}{n}\\
	\end{gather}
	The rest of this proof is very similar and proven in the same ways.This applies for both cases where $\frac{a}{a+b}<\frac{y}{n}$ and $\frac{a}{a+b}\ge\frac{y}{n}$ For sake of time and effort, these details will be omitted. This fact may be more easily proven or thought of intuitively, but this was how I approached the problem.\\
	Thus in the end, we will have achieved
	\begin{gather}
			min\left\{E[\theta],\frac{y}{n}\right\}\le E(\theta | Y = y) \le max\left\{E[\theta],\frac{y}{n}\right\} \square
	\end{gather}
	\item Show that if $P(\theta) = Unif(0,1)$\\
	We have $\Var(\theta|Y=y)\le \Var(\theta)$\\
	Here $\Var(\theta|Y=y)$ is the variance of the distribution of$ P(\theta |Y=y)$ and $\Var(\theta)$ is the variance of the distribution $P(\theta) = Unif(0,1)$
	\begin{gather}
		\Var(\theta)=\frac{1}{12} \text{ Known about the distribution}\\
		\text{Similarly to above, we use bayes rule to find} f(\theta|Y=y)=\frac{P(Y|\theta) * P(\theta)}{P(Y)}\\
		=> f(\theta)|Y=y \propto \binom{n}{y}\theta^y(1-\theta)^{n-y} * 1 \sim Beta(Y,n-Y)\\
		\Var(Beta(Y,n-Y)) = \frac{Y(n-Y)}{(Y+n-Y)^2(Y+n-Y+1)} = \frac{Y(n-Y)}{(n)^2(n+1)}\\
		\text{Now lets analyze the derivate of this variance with respect to Y}\\
		\text{This makes sense as our variance should be some function of Y}\\
		\frac{d}{dy}\frac{Y(n-Y)}{(n)^2(n+1)} \propto n-2y\\
		\text{This function is negative with Y and thus decreasing for all y.}\\
		\text{We can easily see that there is a point of maximum return at } y=\frac{n}{2}\\
		\text{Plug this into our original equation we get }\\
		\Var(Beta(Y,n-Y)) = \frac{Y(n-Y)}{(Y+n-Y)^2(Y+n-Y+1)} =\frac{1}{4(n+1)}\\
		\text{Since n is strictly positive and $y=\frac{n}{2}$, this value is highest with lowest n possible, } n=2\\
		=> max(\Var(Beta(Y,n-Y))) = \frac{1}{4*(2+1)} = \frac{1}{12} \le \Var(\theta) \square
	\end{gather}
\end{enumerate}
\item
Let A, B and C be independent random variables with the following distributions:
\begin{gather}
	P(A = 1) = 0.4; P(A = 2) = 0.6\\
	P(B = -3) = 0.25; P(B = -2) = 0.25; P(B = -1) = 0.25; P(B = 1) = 0.25\\
	P(C = 1) = 0.5; P(C = 2) = 0.4; P(C = 3) = 0.1
\end{gather}
\begin{enumerate}
	\item What is the probability of the quadratic equation $Ax^2+Bx+C=0$ has two real roots that are different?
	\begin{gather}
		\text{we are solving} \frac{-B \pm \sqrt{B^2-4AC}}{2A}\\
		\text{For our roots to be real and different, we require } B^2-4AC > 0 \, \, \, \& B^2-4AC \ne 0\\
		\text{With some thinking, we can see that this is only satisfied in the following cases}\\
		A=1,C=1,B=-3\\
		A=1,C=2,B=-3\\
		A=2,C=1,B=-3\\
	\end{gather}
	If A or C were both 2 or either was great than two, we would only have negative roots as $B^2$ can only take the maximum value of 9 at any time.\\
	Thus we simply add the probabilities of these three joint events
	\begin{gather}
		P(A=1,C=1,B=-3)+P(A=2,C=1,B=-3)+P(A=1,C=2,B=-3)\\
		 = .4*.5*.25+.6*.5*.25+.4*.4*.25 = 0.165
	\end{gather}
	\item What is the probability that the quadratic equation  $Ax^2+Bx+C=0$
	has two real roots and are both strictly positive.
	\begin{gather}
	\text{we are solving} \frac{-B \pm \sqrt{B^2-4AC}}{2A}\\
	\text{For our roots to be real positive, we require } B^2-4AC \ge 0 \, \, \, \& -B - \sqrt{B^2-4AC} > 0\\
	\text{With some thinking, we can see that this is only satisfied in the following cases}\\
	A=1,C=1,B=-2\\
	A=1,C=1,B=-3\\
	A=1,C=2,B=-3\\
	A=2,C=1,B=-3\\
	\end{gather}
	We can see that unlike the last problem, we can allow our roots to be one in the same, so we add another case where $-B - \sqrt{B^2-4AC} = 0$\\
	we get the following
	\begin{gather}
	P(A=1,C=1,B=-3)+P(A=2,C=1,B=-3)+\\
	P(A=1,C=2,B=-3)+P(A=1,C=1,B=-2)\\
	= .4*.5*.25+.6*.5*.25+.4*.4*.25+.4*.5*.25 = 0.215
	\end{gather}
\end{enumerate}
\end{enumerate}
\text{Have a great thanksgiving!}
\end{document}

\documentclass[11pt]{article}
\usepackage{amsgen,amsmath,amstext,amsbsy,amsopn,amssymb}
%\usepackage[dvips]{graphicx,color}
\usepackage{graphicx,color}
\usepackage{graphicx,color,bm}
\usepackage{epsfig}
\usepackage{enumerate}
\usepackage{float}

%\setlength{\oddsidemargin}{0.1 in} \setlength{\evensidemargin}{-0.1
%in} \setlength{\topmargin}{-0.6 in} \setlength{\textwidth}{6.5 in}
%\setlength{\textheight}{8.5 in} \setlength{\headsep}{0.75 in}
%\setlength{\parindent}{0 in} \setlength{\parskip}{0.1 in}

\textwidth 6.3in \textheight 8.8in \topmargin -0.5truein
\oddsidemargin .15truein
\parskip .1in
\renewcommand{\baselinestretch}{1.53}  % double spaced


\newcommand{\homework}[9]{
	\pagestyle{myheadings}
	\thispagestyle{plain}
	\newpage
	\setcounter{page}{1}
	\noindent
	\begin{center}
		\framebox{
			\vbox{\vspace{2mm}
				\hbox to 6.28in { {\bf Math531:~Regression - I  \hfill} }
				\vspace{6mm}
				\hbox to 6.28in { {\Large \hfill #1 (#2)  \hfill} }
				\vspace{6mm}
				\hbox to 6.28in { {\it Instructor: #3 \hfill} }
				\hbox to 6.28in { {\it Office hours: #4  \hfill #6}}
				\vspace{2mm}}
		}
	\end{center}
	\markboth{#1}{#1}
	\vspace*{4mm}
}

% ----------------------- MATH -------------------------
\def\av{\boldsymbol a}
\def\bv{\boldsymbol b}
\def\cv{\boldsymbol c}
\def\dv{\boldsymbol d}
\def\ev{\boldsymbol e}
\def\fv{\boldsymbol f}
\def\gv{\boldsymbol g}
\def\hv{\boldsymbol h}
\def\iv{\boldsymbol i}
\def\gv{\boldsymbol j}
\def\kv{\boldsymbol k}
\def\lv{\boldsymbol l}
\def\mv{\boldsymbol m}
\def\nv{\boldsymbol n}
\def\ov{\boldsymbol o}
\def\pv{\boldsymbol p}
\def\qv{\boldsymbol q}
\def\rv{\boldsymbol r}
\def\sv{\boldsymbol s}
\def\tv{\boldsymbol t}
\def\uv{\boldsymbol u}
\def\vv{\boldsymbol v}
\def\wv{\boldsymbol w}
\def\xv{\boldsymbol x}
\def\yv{\boldsymbol y}
\def\zv{\boldsymbol z}
\def\Av{\boldsymbol A}
\def\Bv{\boldsymbol B}
\def\Cv{\boldsymbol C}
\def\Dv{\boldsymbol D}
\def\Ev{\boldsymbol E}
\def\Fv{\boldsymbol F}
\def\Gv{\boldsymbol G}
\def\Hv{\boldsymbol H}
\def\Iv{\boldsymbol I}
\def\Gv{\boldsymbol J}
\def\Kv{\boldsymbol K}
\def\Lv{\boldsymbol L}
\def\Mv{\boldsymbol M}
\def\Nv{\boldsymbol N}
\def\Ov{\boldsymbol O}
\def\Pv{\boldsymbol P}
\def\Qv{\boldsymbol Q}
\def\Rv{\boldsymbol R}
\def\Sv{\boldsymbol S}
\def\Tv{\boldsymbol T}
\def\Uv{\boldsymbol U}
\def\Vv{\boldsymbol V}
\def\Wv{\boldsymbol W}
\def\Xv{\boldsymbol X}
\def\Yv{\boldsymbol Y}
\def\Zv{\boldsymbol Z}
\def\Abf{\mathbf A}
\def\Bbf{\mathbf B}
\def\Cbf{\mathbf C}
\def\Dbf{\mathbf D}
\def\Ebf{\mathbf E}
\def\Fbf{\mathbf F}
\def\Gbf{\mathbf G}
\def\Hbf{\mathbf H}
\def\Ibf{\mathbf I}
\def\Gbf{\mathbf J}
\def\Kbf{\mathbf K}
\def\Lbf{\mathbf L}
\def\Mbf{\mathbf M}
\def\Nbf{\mathbf N}
\def\Obf{\mathbf O}
\def\Pbf{\mathbf P}
\def\Qbf{\mathbf Q}
\def\Rbf{\mathbf R}
\def\Sbf{\mathbf S}
\def\Tbf{\mathbf T}
\def\Ubf{\mathbf U}
\def\Vbf{\mathbf V}
\def\Wbf{\mathbf W}
\def\Xbf{\mathbf X}
\def\Ybf{\mathbf Y}
\def\Jbf{\mathbf J}
\def\Zbf{\mathbf Z}
\def\Am{\mathrm A}
\def\Bm{\mathrm B}
\def\Cm{\mathrm C}
\def\Dm{\mathrm D}
\def\Em{\mathrm E}
\def\Fm{\mathrm F}
\def\Gm{\mathrm G}
\def\Hm{\mathrm H}
\def\Im{\mathrm I}
\def\Gm{\mathrm J}
\def\Km{\mathrm K}
\def\Lm{\mathrm L}
\def\Mm{\mathrm M}
\def\Nm{\mathrm N}
\def\Om{\mathrm O}
\def\Pm{\mathrm P}
\def\Qm{\mathrm Q}
\def\Rm{\mathrm R}
\def\Sm{\mathrm S}
\def\Tm{\mathrm T}
\def\Um{\mathrm U}
\def\mv{\mathrm V}
\def\Wm{\mathrm W}
\def\Xm{\mathrm X}
\def\Ym{\mathrm Y}
\def\Zm{\mathrm Z}
\newcommand{\Ac}{\mathcal{A}}
\newcommand{\Bc}{\mathcal{B}}
\newcommand{\Cc}{\mathcal{C}}
\newcommand{\Dc}{\mathcal{D}}
\newcommand{\Ec}{\mathcal{E}}
\newcommand{\Fc}{\mathcal{F}}
\newcommand{\Gc}{\mathcal{G}}
\newcommand{\Hc}{\mathcal{H}}
\newcommand{\Ic}{\mathcal{I}}
\newcommand{\Jc}{\mathcal{J}}
\newcommand{\Kc}{\mathcal{K}}
\newcommand{\Lc}{\mathcal{L}}
\newcommand{\Mc}{\mathcal{M}}
\newcommand{\Nc}{\mathcal{N}}
\newcommand{\Oc}{\mathcal{O}}
\newcommand{\Pc}{\mathcal{P}}
\newcommand{\Qc}{\mathcal{Q}}
\newcommand{\Rc}{\mathcal{R}}
\newcommand{\Sc}{\mathcal{S}}
\newcommand{\Tc}{\mathcal{T}}
\newcommand{\Uc}{\mathcal{U}}
\newcommand{\Vc}{\mathcal{V}}
\newcommand{\Wc}{\mathcal{W}}
\newcommand{\Xc}{\mathcal{X}}
\newcommand{\Yc}{\mathcal{Y}}
\newcommand{\Zc}{\mathcal{Z}}
\newcommand{\alphav}{\mbox{\boldmath{$\alpha$}}}
\newcommand{\betav}{\mbox{\boldmath{$\beta$}}}
\newcommand{\gammav}{\mbox{\boldmath{$\gamma$}}}
\newcommand{\deltav}{\mbox{\boldmath{$\delta$}}}
\newcommand{\epsilonv}{\mbox{\boldmath{$\epsilon$}}}
\newcommand{\zetav}{\mbox{\boldmath$\zeta$}}
\newcommand{\etav}{\mbox{\boldmath{$\eta$}}}
\newcommand{\iotav}{\mbox{\boldmath{$\iota$}}}
\newcommand{\kappav}{\mbox{\boldmath{$\kappa$}}}
\newcommand{\lambdav}{\mbox{\boldmath{$\lambda$}}}
\newcommand{\muv}{\mbox{\boldmath{$\mu$}}}
\newcommand{\nuv}{\mbox{\boldmath{$\nu$}}}
\newcommand{\xiv}{\mbox{\boldmath{$\xi$}}}
\newcommand{\omicronv}{\mbox{\boldmath{$\omicron$}}}
\newcommand{\piv}{\mbox{\boldmath{$\pi$}}}
\newcommand{\rhov}{\mbox{\boldmath{$\rho$}}}
\newcommand{\sigmav}{\mbox{\boldmath{$\sigma$}}}
\newcommand{\tauv}{\mbox{\boldmath{$\tau$}}}
\newcommand{\upsilonv}{\mbox{\boldmath{$\upsilon$}}}
\newcommand{\phiv}{\mbox{\boldmath{$\phi$}}}
\newcommand{\varphiv}{\mbox{\boldmath{$\varphi$}}}
\newcommand{\chiv}{\mbox{\boldmath{$\chi$}}}
\newcommand{\psiv}{\mbox{\boldmath{$\psi$}}}
\newcommand{\omegav}{\mbox{\boldmath{$\omega$}}}
\newcommand{\Sigmav}{\mbox{\boldmath{$\Sigma$}}}
\newcommand{\Lambdav}{\mbox{\boldmath{$\Lambda$}}}
\newcommand{\Deltav}{\mbox{\boldmath{$\Delta$}}}
\newcommand{\Omegav}{\mbox{\boldmath{$\Omega$}}}
\newcommand{\varepsilonv}{\mbox{\boldmath{$\varepsilon$}}}

\newcommand{\eps}{\varepsilon}
\newcommand{\epsv}{\mbox{\boldmath{$\varepsilon$}}}

\def\1v{\mathbf 1}
\def\0v{\mathbf 0}
\def\Id{\mathbf I} % identity matrix
\newcommand{\ind}[1]{\mathbbm{1}_{\left[ {#1} \right] }}
\newcommand{\Ind}[1]{\mathbbm{1}_{\left\{ {#1} \right\} }}
\newcommand\indep{\protect\mathpalette{\protect\independenT}{\perp}}\def\independenT#1#2{\mathrel{\rlap{$#1#2$}\mkern2mu{#1#2}}}
\newcommand{\QED}{\begin{flushright} {\bf QED} \end{flushright}}
\newcommand{\R}{\mathbb R}
\newcommand{\Real}{\mathbb R}
\newcommand{\C}{\mathbb C}
\newcommand{\E}{\mathbb E}
\newcommand{\sgn}{\mathop{\mathrm{sign}}}
\def\Pr{\mathrm P}
\def\pr{\mathrm P}
\newcommand{\Var}{\mathop{\rm Var}}
\newcommand{\var}{\mathop{\rm Var}}
\newcommand{\Cov}{\mathop{\rm Cov}}
\newcommand{\cov}{\mathop{\rm Cov}}
\newcommand{\Corr}{\mathop{\rm Corr}}
\newcommand{\ang}{\mathop{\rm Angle}}
\newcommand{\tr}{\mathop{\rm trace}}
\newcommand{\proj}{\mathop{\rm Proj}}
\newcommand{\rank}{\mathop{\rm rank}}

\newcommand{\diag}{\mathop{\rm diag}}
\newcommand{\Diag}{\mathop{\rm diag}}
\newcommand{\sk}{\vspace{0.5cm}}
\newcommand{\ds}{\displaystyle}
\newcommand{\mb}{\mbox}
\newcommand{\wh}{\widehat}
\newcommand{\argmin}{\operatornamewithlimits{argmin}}
\newcommand{\argmax}{\operatornamewithlimits{argmax}}

\newcommand{\norm}[1]{\|#1\|}
\newcommand{\abs}[1]{\left\vert#1\right\vert}
\newcommand{\set}[1]{\left\{#1\right\}}

\newcommand{\To}{\longrightarrow}

\def\equalLaw{\stackrel{\mathcal{L}}{=}}
\def\equallaw{\stackrel{\mathcal{L}}{=}}

\def\half{\frac{1}{2}}

\usepackage{caption}

\begin{document}

\begin{title}
	{\Large\bf Homework 7, DATA 556: Due Tuesday, 11/13/2018}
\end{title}

\author{\bf Alexander Van Roijen}

\maketitle

\newpage
Please complete the following:
\begin{enumerate}
\item Problem 1
Let X and Y be i.i.d. Expo($\lambda$) and T = $\log(X/Y)$. Find CDF and PDF of T
\begin{gather}
	F_t(t) = P(T\le t)=P(\frac{X}{Y}\le e^t) = \int_{0}^{\infty} P(\frac{X}{Y}\le e^t | Y=y) P(Y=y)dy \\
	= \int_{0}^{\infty} P(X \le ye^t | Y=y) f(y)dy =   \int_{0}^{\infty}\int_{0}^{ye^t} f_X(x) f_Y(y)dxdy \\
	= \int_{0}^{\infty} \int_{0}^{ye^t}  \lambda e^{-\lambda x} e^{-\lambda y} dxdy = \int_{0}^{\infty} (1-e^{-\lambda ye^t}) \lambda e^{-\lambda y} dy \\
	= \int_{0}^{\infty} \lambda e^{-\lambda y} dy - \int_{0}^{\infty} e^{-\lambda ye^t} \lambda e^{-\lambda y} dy = 1 - \int_{0}^{\infty} \lambda e^{-\lambda y(e^t +1)} dy = 1 - \frac{1}{e^t +1}e^{-\lambda y(e^t +1)}\big|_0^\infty\\
	= 1 - (0+\frac{1}{e^t +1}) = 1 - \frac{1}{e^t +1}\\
	=> f_T(t) = \frac{d}{dt} (1 - \frac{1}{e^t +1}) = \frac{e^t}{(e^t +1)^2}
\end{gather}
The above used the LOTP as well as properties of independent exponential distributions, what they integrate to, and similar principles. This result is easily verified to hold all the properties necessary of a PDF or CDF.
\item Let X and Y be i.i.d. Expo($\lambda$), and transform them to T = X+Y and W=X/Y
\begin{enumerate}
	\item Find the joint PDF of T and W. Are they independent?
	\begin{gather}
		f_{t,w}=f_{x,y} * \begin{bmatrix}
		\frac{\partial x}{\partial t} & 
		\frac{\partial x}{\partial w} \\
		\frac{\partial y}{\partial t} & 
		\frac{\partial y}{\partial w} & 
		\end{bmatrix}
		= f_x(x) * f_y(y) \begin{bmatrix}
		\frac{1}{1+t} & 
		\frac{t}{1+w} - \frac{tw}{(1+w)^2} \\
		\frac{1}{1+w} & 
		\frac{-t}{(1+w)^2} & 
		\end{bmatrix}\\
		= \lambda^2e^{-\lambda(x+y)} |\frac{-tw}{(1+w)^3} - \frac{t(1+w)}{(1+w)^3} + \frac{tw}{(1+w)^3}| = \lambda^2e^{-\lambda t}\frac{t(1+w)}{(1+w)^3}\\
		=\lambda^2te^{-\lambda t}\frac{1}{(1+w)^2}
	\end{gather}
	Thus they are independent as we are able to separate their joint distribution into two functions of their respective variables in the form of $g(t)*h(w)$
	\begin{gather}
		g(t) = \lambda^2te^{-\lambda t} \, \, \, \& h(w) = \frac{1}{(1+w)^2}
	\end{gather}
	\item Find the marginal PDFs of T and W.
	\begin{gather}
		f(t) = \int_{-\infty}^{\infty} f(t,w) dw = \int_{-\infty}^{\infty} \lambda^2te^{-\lambda t}\frac{1}{(1+w)^2} dw \text{ w/} 0\le w\\
		\text{This is obvious as } W=\frac{X}{Y} \text{ and } X\ge0Y\ge0\\
		=>\int_{0}^{\infty} \lambda^2te^{-\lambda t}\frac{1}{(1+w)^2} dw = \lambda^2te^{-\lambda t} \frac{-1}{(1+w)} \big{|}_0^\infty = \lambda^2te^{-\lambda t} \\
		f(w) = \int_{-\infty}^{\infty} f(t,w) dt =  \int_{0}^{\infty} \lambda^2te^{-\lambda t}\frac{1}{(1+w)^2} dw \text{ w/} 0\le t\\
		\text{This is similarly obvious as } T = X+Y\text{ and } X\ge0Y\ge0\\
		=> \frac{1}{(1+w)^2} -e^{-\lambda t}(\lambda t + 1)\big{|}_0^\infty = \frac{1}{(1+w)^2} (0 +  1(0+1)) = \frac{1}{(1+w)^2} \\
		\text{The first term is evaluated to zero as } e^{\lambda t} \text{ grows faster than } \lambda t
	\end{gather}
	This confirms our previous result that they are independent as $f_t(t) * f_w(w) = f_{w,t}(w,t)$
\end{enumerate}
\item Let U $\sim$ Unif(0,1) and X $\sim$ Expo($\lambda$), independently. Find the PDF of U+X \\
We have $0 \le U \le 1$ for U and $0\le X \le \infty$
\begin{gather}
	T=U+X=>f(t)=\int_{-\infty}^{\infty}f_x(x)f_u(t-x)dx \\
	\text{We require that } x\ge0 \& \, 0\le t-x\le 1 => t\ge x\ge 1-t \text{ By the properties of our distribs} \\
	\text{Thus we need to match all properties and get the following}\\
	\int_{max(0,t-1)}^{t}f_x(x)f_u(t-x)dx => \int_{0}^{t}f_x(x)f_u(t-x)dx = \int_{0}^{t} \lambda e^{-\lambda(x)} dx = 1- e^{-\lambda(t)}\\
	=>\int_{t-1}^{t}f_x(x)f_u(t-x)dx => \int_{t-1}^{t}f_x(x)f_u(t-x)dx = \int_{t-1}^{t} \lambda e^{-\lambda(x)} dx = e^{-\lambda(t-1)}- e^{-\lambda t} = e^{-\lambda t}(e^\lambda -1)\\
	f(t)=\begin{cases}
	e^{-\lambda t}(e^\lambda -1) & t > 1 \\
	1- e^{-\lambda(t)} & 0\le t \le 1 \\
	0 &t<0
	\end{cases}
\end{gather}
\item Let X and Y be i.i.d. Expo($\lambda$). Use a convolution integral to show that the PDF of L = X - Y is
\begin{gather}
	f_L(l) = \frac{\lambda}{2}e^{-\lambda|l|}\\
	\text{Note that } L = X + (-Y)\\
	\text{Further, we  know } x-l\ge0 => x\ge l \text{ if } L<0
	\text{ we get the following }\\
	f_L(l) = \int_{0}^{\infty}f_x(x)f_{-y}(l-x)dx =  \int_{0}^{\infty}f_x(x)f_{y}(x-l)dx \text{ This is easily seen as }\\
	f_{-y}(t-x) = f_{y}(-1(t-x)) \text{ as }g^{-1}(-y)=y \text{ as } g(x) = -x \text{ and } |\frac{d}{dy}(-y)| = 1\\
	=\int_{0}^{\infty}\lambda^2 e^{-\lambda(2x-l)}dx = e^{\lambda l}\int_{0}^{\infty}\lambda e^{-\lambda(2x)}dx = e^{\lambda l} * -\frac{\lambda e^{-\lambda 2x}}{2}\big{|}_0^\infty =  e^{\lambda l} * -\frac{\lambda e^{-\lambda 2x}}{2}\big{|}_0^\infty = e^{\lambda l} * (0+\frac{\lambda}{2}) = \frac{\lambda e^{\lambda l}}{2}\\
	\text{Now for } l\ge 0\\
	f_L(l) = \int_{l}^{\infty}f_x(x)f_{-y}(l-x)dx =  \int_{l}^{\infty}f_x(x)f_{y}(x-l)dx \\
	=\int_{l}^{\infty}\lambda^2 e^{-\lambda(2x-l)}dx = e^{\lambda l}\int_{l}^{\infty}\lambda e^{-\lambda(2x)}dx = e^{\lambda l} * -\frac{\lambda e^{-\lambda 2x}}{2}\big{|}_l^\infty =  e^{\lambda l} * -\frac{\lambda e^{-\lambda 2x}}{2}\big{|}_l^\infty \\
	= e^{\lambda l} * (0+\frac{\lambda}{2}e^{-\lambda 2l}) = \frac{\lambda e^{-\lambda l}}{2} \\
	\text{we get  }
	f_L(l)=
	\begin{cases}
		\frac{\lambda e^{-\lambda l}}{2} & l \ge 0 \\
		\frac{\lambda e^{\lambda l}}{2} & l < 0\\
	\end{cases}
	\text{which is equivalent to saying } 	
	f_L(l) = \frac{\lambda}{2}e^{-\lambda|l|} \square
\end{gather}
\item Use a convolution integral to show that if X $\sim N(\mu_1,\sigma)$ and Y $\sim N(\mu_2,\sigma)$ are independent, then
\begin{gather}
	T=X+Y\sim N(\mu_1+\mu_2,2\sigma^2)
\end{gather}
\begin{gather}
	f(t)=\int_{-\infty}^{\infty}f_x(x)f_y(t-x)dx = \int_{-\infty}^{\infty} \frac{1}{\sqrt{2\pi}\sigma}e^{\frac{-1((t-x)-\mu_2)^2}{2\sigma^2}} \frac{1}{\sqrt{2\pi}\sigma}e^{\frac{-1(x-\mu_1)^2}{2\sigma^2}}dx\\
	=\int_{-\infty}^{\infty} \frac{1}{\sqrt{2\pi}\sqrt{2\pi}\sigma^2}e^{-1\frac{(t-x-\mu_2)^2+(x-\mu_1)^2}{2\sigma^2}}dx\\
	=\int_{-\infty}^{\infty} \frac{1}{\sqrt{2\pi}\sqrt{2\pi}\sigma^2}e^{-1\frac{t^2+x^2+\mu_2^2-2tx-2t\mu_2+2x\mu_2+x^2-2\mu_1x+\mu_1^2}{2\sigma^2}}dx\\
	=\int_{-\infty}^{\infty} \frac{1}{\sqrt{2\pi}\sqrt{2\pi}\sigma^2}e^{-1\frac{2x^2 -2x(t-\mu_2+\mu_1) + t^2 + \mu_2^2 -2t\mu_t +\mu_1^2}{2\sigma^2}}dx\\
	=\int_{-\infty}^{\infty} \frac{1}{\sqrt{2\pi}\sqrt{2\pi}\sigma^2}e^{-1\frac{x^2 -x(t-\mu_2+\mu_1) + \frac{1}{2}(t^2 + \mu_2^2 -2t\mu_t +\mu_1^2)}{\sigma^2}}dx\\
	=\int_{-\infty}^{\infty} \frac{1}{\sqrt{2\pi}\sqrt{2\pi}\sigma^2}e^{-1\frac{(x-\frac{t-\mu_2+\mu_1}{2})^2 - \frac{(t-\mu_2+\mu_1)^2}{4} + \frac{(t-\mu_2)^2 +\mu_1^2}{2}}{\sigma^2}}dx
\end{gather}
\item Let $W_1, W_2$ be two R.V. with joint distrib\\
$P(W_1\le w_1, W_2 \le w_2) = \int_{-\infty}^{w_1}\int_{-\infty}^{w_2} \frac{1}{2\pi}e^{\frac{-1(x^2+y^2)}{2}}dxdy$\\
consider two other RVs $Z_1 = |W_1|$ and $Z_2=|W_2|$. In words $Z_1$ is the absolute value of $W_1$ and similar for $Z_2$
\begin{enumerate}
	\item Show $Z_1$ independent of $Z_2$
	\\
	Note that it is very easy to see that $W_1 \, \& \, W_2 \sim N(0,1)$ i.i.d.
	\begin{gather}
		\int_{-\infty}^{w_1}\int_{-\infty}^{w_2} \frac{1}{2\pi}e^{\frac{-1(x^2+y^2)}{2}}dxdy = \int_{-\infty}^{w_1}\frac{1}{\sqrt{2\pi}}e^{\frac{-1(y^2)}{2}}dy\int_{-\infty}^{w_2} \frac{1}{\sqrt{2\pi}}e^{\frac{-1(x^2)}{2}}dx =\phi(w_1)\phi(w_2)\\
	\end{gather}
	We further know that functions of independent random variables are independent, and thus $Z_1 \, \& \, Z_2$ are independent
	\item Show that $Z_1$ and $Z_2$ have the same distribution, and find it
	\begin{gather}
		P(Z_1\le z_1) = P(|W_1|\le z_1) = P( -z_1 \le W_1 \le z_1) = \varphi(z_1) - \varphi(-z_1) = \varphi(z_1) - (1- \varphi(z_1))\\
		= \int_{0}^{z_1}\frac{1}{\sqrt{2\pi}}(e^\frac{-1*w_1^2}{2} + e^\frac{-1*w_1^2}{2})dw_1 \, \, \, z_1>0
	\end{gather}
	The same is trivially done for $Z_2$. Now lets analyze what kind of distribution this is.\\
	It is clear that the distributions are the same, as they both are composed of a sum of two normal distributions over the same range of values $0 \rightarrow \infty$.\\
	These both also are strictly increasing and sum to one.\\
	\begin{gather}
		\int_{0}^{\infty}\frac{1}{\sqrt{2\pi}}(e^\frac{-1*w_1^2}{2})dw_1 + \int_{0}^{\infty}\frac{1}{\sqrt{2\pi}}(e^\frac{-1*w_1^2}{2})dw_1 = \frac{1}{2} + \frac{1}{2} = 1
	\end{gather} Thinking about these random variables, since they are the absolute value of a normal distribution, which is symmetric around mean 0, it appears to reflect the negative area back on to its positive area. Effectively "doubling up" the area under our curve.
	\begin{gather}
	\frac{1}{\sqrt{2\pi}}(e^\frac{-1*w_1^2}{2}) + \frac{1}{\sqrt{2\pi}}(e^\frac{-1*w_1^2}{2}) =
		\frac{2}{\sqrt{2\pi}} e^\frac{-1*w_1^2}{2} \, \, \, w_1>0
	\end{gather}
	Thus we have two RVs that follow the same "double positive" normal distribution with $N(0,1)$
\end{enumerate}
\end{enumerate}

\end{document}
